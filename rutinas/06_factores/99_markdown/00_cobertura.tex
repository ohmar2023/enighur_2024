% Options for packages loaded elsewhere
\PassOptionsToPackage{unicode}{hyperref}
\PassOptionsToPackage{hyphens}{url}
%
\documentclass[
]{article}
\author{}
\date{\vspace{-2.5em}}

\usepackage{amsmath,amssymb}
\usepackage{lmodern}
\usepackage{iftex}
\ifPDFTeX
  \usepackage[T1]{fontenc}
  \usepackage[utf8]{inputenc}
  \usepackage{textcomp} % provide euro and other symbols
\else % if luatex or xetex
  \usepackage{unicode-math}
  \defaultfontfeatures{Scale=MatchLowercase}
  \defaultfontfeatures[\rmfamily]{Ligatures=TeX,Scale=1}
\fi
% Use upquote if available, for straight quotes in verbatim environments
\IfFileExists{upquote.sty}{\usepackage{upquote}}{}
\IfFileExists{microtype.sty}{% use microtype if available
  \usepackage[]{microtype}
  \UseMicrotypeSet[protrusion]{basicmath} % disable protrusion for tt fonts
}{}
\makeatletter
\@ifundefined{KOMAClassName}{% if non-KOMA class
  \IfFileExists{parskip.sty}{%
    \usepackage{parskip}
  }{% else
    \setlength{\parindent}{0pt}
    \setlength{\parskip}{6pt plus 2pt minus 1pt}}
}{% if KOMA class
  \KOMAoptions{parskip=half}}
\makeatother
\usepackage{xcolor}
\IfFileExists{xurl.sty}{\usepackage{xurl}}{} % add URL line breaks if available
\IfFileExists{bookmark.sty}{\usepackage{bookmark}}{\usepackage{hyperref}}
\hypersetup{
  hidelinks,
  pdfcreator={LaTeX via pandoc}}
\urlstyle{same} % disable monospaced font for URLs
\usepackage[margin=1in]{geometry}
\usepackage{longtable,booktabs,array}
\usepackage{calc} % for calculating minipage widths
% Correct order of tables after \paragraph or \subparagraph
\usepackage{etoolbox}
\makeatletter
\patchcmd\longtable{\par}{\if@noskipsec\mbox{}\fi\par}{}{}
\makeatother
% Allow footnotes in longtable head/foot
\IfFileExists{footnotehyper.sty}{\usepackage{footnotehyper}}{\usepackage{footnote}}
\makesavenoteenv{longtable}
\usepackage{graphicx}
\makeatletter
\def\maxwidth{\ifdim\Gin@nat@width>\linewidth\linewidth\else\Gin@nat@width\fi}
\def\maxheight{\ifdim\Gin@nat@height>\textheight\textheight\else\Gin@nat@height\fi}
\makeatother
% Scale images if necessary, so that they will not overflow the page
% margins by default, and it is still possible to overwrite the defaults
% using explicit options in \includegraphics[width, height, ...]{}
\setkeys{Gin}{width=\maxwidth,height=\maxheight,keepaspectratio}
% Set default figure placement to htbp
\makeatletter
\def\fps@figure{htbp}
\makeatother
\setlength{\emergencystretch}{3em} % prevent overfull lines
\providecommand{\tightlist}{%
  \setlength{\itemsep}{0pt}\setlength{\parskip}{0pt}}
\setcounter{secnumdepth}{-\maxdimen} % remove section numbering
\usepackage{xcolor}
\usepackage{graphicx}
\usepackage{eso-pic}
\definecolor{aliceblue}{rgb}{0.94, 0.97, 1.0}
\definecolor{trueblue}{rgb}{0.0, 0.45, 0.81}
\definecolor{tuftsblue}{rgb}{0.28, 0.57, 0.81}
\ifLuaTeX
  \usepackage{selnolig}  % disable illegal ligatures
\fi

\begin{document}

\pagecolor{aliceblue}

\AddToShipoutPictureBG*{
  \put(0,0){
    \parbox[b][\paperheight]{\paperwidth}{
      \vfill
      \centering
      \includegraphics[width=18cm,height=\paperheight,keepaspectratio]{99_imagenes/01_fondo/fondo.png}
      \vfill
    }
  }
}

\thispagestyle{empty}

\begin{titlepage}
  \begin{center}
    \vspace*{5cm}
    
    {\Huge\bfseries \textcolor{trueblue}{INFORME DE COBERTURA}   \par}
    \vspace{0.2cm}
    
    {\Huge\bfseries \textcolor{trueblue} {ENIGHUR} \par}
    \vspace{0.2   cm}
    
    {\Large\bfseries \textcolor{trueblue} {Periodo de referencia: VII} \par}
    \vspace{1cm}
    
    {\Large  \textcolor{trueblue}{Dirección de Infraestructura Estadística y Muestreo \\ 
    Gestión de Diseño Muestral}
    \par}
    \vspace{0.25cm}
    
    {\large  \textcolor{trueblue}{Omar LLambo} \par}
    \vspace{0.25cm}
    
    %{\Large \textcolor{trueblue}{Julio 2025 } \par}
    
    \vfill
    
    {\large \textcolor{trueblue} {Julio - 2025}\par}
    {\large \textcolor{trueblue} {Quito - Ecuador}\par}
  \end{center}
\end{titlepage}


\newpage

\textcolor{trueblue}{\section{Introducción}}

Con los resultados de las entrevistas realizadas en las viviendas
seleccionadas en la muestra, se analiza la cobertura de la encuesta que
interfiere directamente con los procesos relacionados al diseño
muestral. La cobertura se refiere a la información que se obtuvo en
campo y que ha sido proporcionada por los informantes de los hogares.

Se presentan los resultados acorde a la información recolectada desde el
periodo 01 hasta el periodo 07. A lo largo del documento se hace
referencia y se presentan los resultados desagregados por provincia y
por Coordinación Zonal: Administración Central Campo (Adm.C.Campo),
Centro, Litoral y Sur.

\textcolor{trueblue}{\section{Objetivos}}

\begin{itemize}
\item
  Evaluar la cobertura de cada periodo considerando cada dominio de
  estudio y con ello identificar los puntos críticos que requieren una
  especial atención de monitoreo.
\item
  Identificar las novedades que se generan posterior a la selección de
  la segunda etapa de muestreo (selección de viviendas).
\item
  Detallar las novedades relacionadas a los cambios cartográficos y
  categorización de la condición de ocupación de las viviendas.
\end{itemize}

\textcolor{trueblue}{\section{Descripción básica del Diseño Muestral}}

El diseño muestral implementado en la Encuesta Nacional de Ingresos de
Hogares Urbanos y Rurales (ENIGHUR) es un muestreo probabilístico
bietápico estratificado de elementos. En la primera etapa, se selecciona
una muestra estratificada de UPM con probabilidad proporcional al tamaño
(PPT), donde la medida de tamaño de cada UPM está dada por el total de
viviendas particulares ocupadas. Luego, se enlista la totalidad de
viviendas que conforman cada UPM para, en una segunda etapa, seleccionar
aleatoriamente un total fijo de 12 viviendas en cada UPM seleccionada.

Entre las principales características de la encuesta se presentan las
siguientes:

\begin{itemize}
\tightlist
\item
  La ENIGHUR tiene un tamaño muestral total de 3432 UPM distribuidas
  espacial y temporalmente.
\item
  La encuesta está planificada a lo largo de trece periodos: Se levantan
  264 UPM por periodo.
\item
  Cada periodo está constituido por cuatro semanas: En cada semana se
  levantan 66 UPM.
\end{itemize}

\textcolor{trueblue}{\section{Control de cobertura de campo y muestral}}

La cobertura hace referencia a la información obtenida en campo y
proporcionada por los informantes de las viviendas seleccionadas. En el
presente documento se realiza el seguimiento de la cobertura de acuerdo
a los diferentes cortes de base de datos.

En este sentido, la cobertura a nivel de UPM, vivienda y población
objetivo toma en cuenta el resultado de la entrevista y la condición de
ocupación de las viviendas visitadas, las cuales se clasifican en dos
grandes grupos: elegibilidad conocida y elegibilidad desconocida, con la
siguiente subclasificación:

\begin{itemize}
\item
  \textbf{Elegibilidad conocida}

  \begin{itemize}
  \tightlist
  \item
    Elegible respondiente (RE)

    \begin{itemize}
    \tightlist
    \item
      Efectiva
    \end{itemize}
  \item
    Elegible no respondiente (NR)

    \begin{itemize}
    \tightlist
    \item
      Rechazo
    \end{itemize}
  \item
    No elegibles (NE)

    \begin{itemize}
    \tightlist
    \item
      Temporal
    \item
      Desocupada
    \item
      En construcción
    \item
      Inhabitable o destruída
    \item
      Convertida en negocio
    \item
      Otra razón, cuál?
    \end{itemize}
  \end{itemize}
\item
  \textbf{Elegibilidad desconocida}

  \begin{itemize}
  \tightlist
  \item
    Elegibilidad desconocida (ED)

    \begin{itemize}
    \tightlist
    \item
      Nadie en casa
    \end{itemize}
  \end{itemize}
\end{itemize}

\hypertarget{tasas-de-conformidad}{%
\subsection{Tasas de conformidad}\label{tasas-de-conformidad}}

Tomando en cuenta tanto la clasificación como la subclasificación de las
viviendas investigadas, se calculan las siguientes tasas de conformidad:

\begin{itemize}
\tightlist
\item
  Tasa de respondientes (\(T_{RE}\))
\end{itemize}

Esta tasa permite conocer la proporción de viviendas efectivas para el
total de viviendas visitadas.

\[T_{RE} = \frac{RE}{RE + NR + NE + ED}\]

\begin{itemize}
\tightlist
\item
  Tasa de no respondientes (\(T_{NR}\))
\end{itemize}

Esta tasa permite conocer la proporción de viviendas elegibles no
efectivas para el total de viviendas visitadas.

\[T_{NR} = \frac{NR}{RE + NR + NE + ED}\]

\begin{itemize}
\tightlist
\item
  Tasa de no elegibles (\(T_{NE}\))
\end{itemize}

Esta tasa permite conocer la proporción de viviendas no elegibles que
formaron parte de la muestra y que, en un principio, no deberían haber
formado parte del marco de muestreo.

\[T_{NE} = \frac{NE}{RE + NR + NE + ED}\]

\begin{itemize}
\tightlist
\item
  Tasa de elegibilidad desconocida (\(T_{ED}\)).
\end{itemize}

Esta tasa permite conocer la proporción de viviendas cuya condición de
elegibilidad no pudo ser determinada en campo.

\[T_{ED} = \frac{ED}{RE + NR + NE + ED}\] A continuación se presenta el
análisis de las tasas de conformidad a nivel de UPM, vivienda y
población objetivo. \newpage

\textcolor{trueblue}{\section{Control de cobertura - Acumulada}}

En la Tabla 1 podemos apreciar el detalle de la condición de ocupación y
el resultado de la entrevista. Esta información constituye el insumo
para el cálculo de las tasas de conformidad. La información se presenta
desagregada a nivel de coordinación zonal y condición de ocupación.

\[\textbf{Tabla 1:}\text{ Condición de ocupación y resultado de entrevista}\]
\vspace{-8truemm}

\begin{longtable}[]{@{}lrrrr@{}}
\toprule
Condición Ocupación & ADM. C. CAMPO & CENTRO & LITORAL & SUR \\
\midrule
\endhead
Completa & 4214 & 4351 & 5457 & 5013 \\
Nadie en casa & 210 & 65 & 297 & 164 \\
Otra razón & 19 & 10 & 100 & 54 \\
Rechazo & 445 & 107 & 477 & 252 \\
Rechazo a mitad de la encuesta & 12 & 38 & 58 & 75 \\
Vivienda convertida en negocio & 8 & 1 & 10 & 3 \\
Vivienda desocupada & 95 & 47 & 156 & 74 \\
Vivienda en construcción & 1 & 4 & 26 & 6 \\
Vivienda inhabitada o destruida & 0 & 2 & 13 & 2 \\
Vivienda temporal & 57 & 88 & 170 & 91 \\
NA & 0 & 0 & 2 & 0 \\
\bottomrule
\end{longtable}

\vspace{-1truemm}

A continuación podemos evidenciar que la cobertura nacional está sobre
el umbral establecido, sin embargo, a lo largo del informe podemos
evidenciar las deficiencias de cobertura que se alcanza a otros niveles
de desagregación.

\[\textbf{Gráfico 1:}\text{ Cobertura nacional acumulada}\]
\includegraphics{00_cobertura_files/figure-latex/unnamed-chunk-3-1.pdf}

Desagregando los resultados para cada una de las coordinaciones zonales
podemos evidenciar que las que presentan más inconvenientes con respecto
a ``rechazos'' son la coordinación Litoral y Adm.C.Campo.

\[\textbf{Gráfico 2:}\text{ Cobertura nacional por zonal acumulada}\]
\includegraphics{00_cobertura_files/figure-latex/unnamed-chunk-4-1.pdf}

\newpage
\vspace{-1truemm}

\hypertarget{tasas-de-conformidad-por-periodo}{%
\subsection{Tasas de conformidad por
periodo}\label{tasas-de-conformidad-por-periodo}}

Se presenta la información de las tasas de conformidad en cada periodo
(periodo 01 - periodo 07). Podemos evidenciar que los periodos más
deficientes a nivel de cobertura son el 01 y el 02. El periodo 07
presenta un retroceso con respecto a la calidad de cobertura de los dos
periodos que lo preceden.

\[\textbf{Gráfico 3:}\text{ Tasa de conformidad por periodo}\]
\includegraphics{00_cobertura_files/figure-latex/unnamed-chunk-5-1.pdf}

\hypertarget{tasas-de-conformidad-por-provincia}{%
\subsection{Tasas de conformidad por
provincia}\label{tasas-de-conformidad-por-provincia}}

Con el objetivo de evaluar el comportamiento de las tasas de conformidad
por provincia, se presenta el gráfico 4 en el que podemos apreciar que
la provincia de Pichincha, Galápagos y Santo Domingo se mantienen por
debajo del umbral establecido. La provincia de Guayas, a pesar de estar
sobre el umbral, requiere la misma atención que las otras provincias
mencionadas.

\[\textbf{Gráfico 4:}\text{ Tasa de conformidad por provincia}\]
\includegraphics{00_cobertura_files/figure-latex/unnamed-chunk-6-1.pdf}
\newpage

\hypertarget{tasas-de-conformidad-a-nivel-de-upm}{%
\subsection{Tasas de conformidad a nivel de
UPM}\label{tasas-de-conformidad-a-nivel-de-upm}}

El presente apartado reúne la información de las tasas de conformidad a
nivel de UPM, es decir, se evalúa los umbrales en los que se encuentran
y se identifica la manera en la que están conformadas las UPM de acuerdo
al resultado de la encuesta de cada una de las viviendas que la
conforman.

En la siguiente tabla se presentan los resultados desagregados por
coordinación zonal.

\[\textbf{Tabla 2: }\text{Tasas de conformidad a nivel de UPM}\]

\begin{longtable}[]{@{}llll@{}}
\toprule
zonal & Entre 50\% - 75\% & Mayor a 75\% & Menor a 50\% \\
\midrule
\endhead
ADM. C. CAMPO & 10.95\% & 82.14\% & 6.9\% \\
CENTRO & 1.53\% & 98.47\% & 0.00\% \\
LITORAL & 17.14\% & 76.25\% & 6.61\% \\
SUR & 8.4\% & 89.5\% & 2.1\% \\
\bottomrule
\end{longtable}

\[\textbf{Gráfico 5: }\text{Tasas de conformidad a nivel de UPM}\]
\includegraphics{00_cobertura_files/figure-latex/unnamed-chunk-8-1.pdf}

\newpage
\textcolor{trueblue}{\section{Evaluación de la efectividad por UPM }}

En el presente apartado se evalúa la UPM como ``Efectiva'' y ``No
efectiva''. De esta manera podemos evaluar por zonal el comportamiento
en estas dos categorías. También es de interés detectar una posible
acumuluación geográfica de aquellas UPM catalogadas como ``No
efectivas''.

\includegraphics{00_cobertura_files/figure-latex/unnamed-chunk-9-1.pdf}

\includegraphics{00_cobertura_files/figure-latex/unnamed-chunk-10-1.pdf}

\includegraphics{00_cobertura_files/figure-latex/unnamed-chunk-11-1.pdf}

\includegraphics{00_cobertura_files/figure-latex/unnamed-chunk-12-1.pdf}

\textcolor{trueblue}{\section{Evaluación del levantamiento de la muestra}}

Con el objetivo de evidenciar el avance del levantamiento a nivel de
UPM, se muestra el siguiente gráfico que resume las UPM visitadas y las
que están por visitar. Periodo de referencia: 7.

\[\textbf{Gráfico 5:}\text{ Evaluación del levantamiento de la muestra}\]
\includegraphics{00_cobertura_files/figure-latex/unnamed-chunk-13-1.pdf}

\end{document}
